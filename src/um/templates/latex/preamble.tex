\documentclass{article}

%# Encodings
\usepackage[utf8]{inputenc}
\usepackage[T1]{fontenc}

%# Language
\usepackage[german]{babel}

%# Maths
\usepackage{amssymb,amsmath}

%# Tables
\usepackage{longtable,booktabs}

%# Margins
\usepackage{geometry}
\geometry{
  paper=a4paper,
  width=6.5in,    %# .content {width: 7.0in, margin-left: auto, margin-right: auto}
  top=0.9in,      %# .content {margin-top: 0.9in}
  bottom=0.7in,   %# .content {margin-bottom: 0.9in}
  footskip=0.7in, %# footer {margin-top: 50px}
  includefoot,    %# include the footer into calculation of total text height
}

%# Paragraph indentation
\setlength{\parindent}{0em}
%# Paragraph spacing
\setlength{\parskip}{1.0em}
%# line spacing
\onehalfspacing

%# Font
\renewcommand{\familydefault}{\sfdefault}

%# Headers and footers
%# cf. https://www.sharelatex.com/learn/Headers_and_footers
\usepackage{fancyhdr}
\pagestyle{fancy}
\fancyhf{}  %# clear all headers and footers

\renewcommand{\headrule}{}  %# header {border: none}
\renewcommand{\footrulewidth}{.5pt}  %# footer {border-top: 0.5px solid}
\renewcommand{\footruleskip}{12pt}  %# footer {padding-top: 20px}


\lfoot{\small Aufgabe \VAR{ exercise.slug } von \VAR{ exercise.author.name } / \href{ \VAR{ exercise.license_url}}{\VAR{ exercise.license_name_long }}.
Aufgabenblatt erstellt mit \href{https://www.unkenmathe.de}{unkenmathe.de}.}  %# footer left

%# Cross-references
\usepackage{hyperref}
\hypersetup{
    colorlinks=true,
    pdftitle={Aufgabe},
    pdfauthor={Unkenmathe}
}
